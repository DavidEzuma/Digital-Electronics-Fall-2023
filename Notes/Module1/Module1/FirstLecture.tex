\documentclass[a4paper, 12pt]{article}

\usepackage[margin=1in]{geometry}

\title{First day Lecture}
\date{8/21/2023}
\author{Dave Ezuma}

\begin{document}
    \maketitle
    \section*{Intro}
    \begin{itemize}
        \item Understnad devices and how they are used to create digital circuits
        \item Design techniques for digital circuits
        \item Testing and implementing digital desings onto physical hardware
    \end{itemize}
    \section{Digital Overview}
    \textbf{Digital} refers to the representation of an element with a set of \textit{discrete} values. \\
    \texttt{Analog} represents an element as a set of continous values. To give an example, a sine wavere
    presents a smooth and conitnous (digital) course of motion

    \subsection*{Digital size}
    Digital information size is determinde by the number of bits, an each bit represents a digit for the digital information.
    The more digits, the larger the element can be represented, or greater precision.
    \begin{itemize}
        \item An analog value represents with 10 $\rightarrow$ bits will have less precision than with 12 bits
    \end{itemize}
    \
    \subsection*{Advantages of digital}
    \begin{itemize}
        \item Easier to use for computing \& storing data
        \item Digital signals are less prone to noise effects
        \item Fast data transmission with ability to encrypt, modulate, etc.
        \item multiple functionality in a small form factor device
    \end{itemize}

    \subsection*{Started with Analog}
    \begin{itemize}
        \item Transistor is base for digital electronics: 
        Analog devices with operates
        as an ON/OFF switch
        \item an input voltage swithces the output voltage ON of OFF
        \item Output voltage of transistor is 0 V or a specific voltage (Vcc, Vdd) $\rightarrow$
    therefore, value is digital output value is also binary
    \end{itemize}
    Early logic chips used BJT (bipolar junction) transistors that operated at 5 V. Known as TTL logic (transistor $\rightarrow$ transistor logic)

    \subsection*{MOS Technology}
    \texttt{MOSFET} (metal oxide semiconductor field effect) revolutionized digital logic by
    allowing miniaturization, high density, and lower power consumption.
    \par ovetime, transistors were able to be manufactured ever smaller.\\ \textbf{Moores law:} 
    Density of transistors on a chip doubles about every 2 years. Leads to
    increased performance, capacity, functionality, etc.

    \subsection{digital Voltage}
    Digital voltage levels can be described by different terms
    \begin{itemize}
        \item Low high
        \item 0,1
        \item False, True
        \item Positive logic, Negative Logic
    \end{itemize}
    The high voltage  level is typically the operating voltage of the logic device

    \subsection*{Why Digital Voltage}
    Digital voltage is analogous to a “yes” or “no” for a logical expression. don't want "maybe"
     Logical expressions are implemented in logic circuits.
     \par Logic circuits are the next step in building functions for a digital system (computers).
     Transistors turning on and off is how a digital system works, only understanding the concept 
     of high/low, true/false, 1\& 0 etc.
1
     \subsection*{Implementing Digital Circuit}
\end{document}