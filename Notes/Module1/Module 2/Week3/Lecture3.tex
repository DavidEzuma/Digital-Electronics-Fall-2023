\documentclass[a4paper,12pt]{article}

\usepackage[margin=1in]{geometry}
\usepackage{tikz}
\usepackage{amssymb}
\usepackage{xcolor}
\usepackage{circuitikz}

\renewcommand{\rightarrow}{\ra}
\newenvironment{minipage6cm}{
  \begin{minipage}{6cm}
}{
  \end{minipage}
}

\title{\texttt{Binary Math}\\\hrulefill}
\author{digital electronics}
\date{\small{9/6/2023}}

\begin{document}
    \maketitle

    \subsubsection*{Decimal Number Range}
        The range of decimal numbers that can be represented in binary in dependent on whether it's signed or unsinged.The range of decimal numbers for unsigned binary is from 0 to $2^n-1$. for signed binary, it would be from $-(-2^{n-1} + 1)$ to $(-2^{n-1} + 1)$
        \begin{itemize}
            \item Using 2s compliment, an additional Negative value can be represented with any given number of bits.
        \end{itemize}
\end{document}