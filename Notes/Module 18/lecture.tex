\documentclass[a4paper,12pt]{article}

\usepackage[margin=1in]{geometry}
\usepackage{tikz}
\usepackage{amssymb}
\usepackage{xcolor}
\usepackage{circuitikz}
\usepackage{graphicx}

\newcommand{\ra}{$\rightarrow$}
\newenvironment{6mini}{
  \begin{minipage}{6cm}
}{
  \end{minipage}
}

\title{\texttt{One Hot Encoding}\\\hrulefill}
\author{Module 18 :)}
\date{\small{12/4/2023}}

\begin{document}
    \maketitle

    \subsubsection{Sequential Design So Far}
        Designing circuits sof ar has been achieved through Binary Encoding
        \begin{itemize}
            \item Each state is a given binar number
            \item Use state table and k-maps to get simplified logic circuits for next states and output
        \end{itemize}
        This is limited by the number of total states and inputs \ra~K-map becomes too large to handle by hand. 
    \section{One Hot Encoding}
    Instead of assigning each state a binary number, \underline{each state is represented as a single binary value}.
    \begin{itemize}
        \item Number of bits in the value is the number of states
        \item only one bit can be 1 for each value
        \item 1 flip flop for each state
        \item One hot encoding sues more flip flops than binary Encoding
        \item No state table necessary
    \end{itemize}
    *For 3 states, the encoding is State 0 -$>$001, State 1 -$>$ 010, State 2 -$>$100
\end{document}