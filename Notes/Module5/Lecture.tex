\documentclass[a4paper,12pt]{article}

\usepackage[margin=1in]{geometry}
\usepackage{tikz}
\usepackage{amssymb}
\usepackage{xcolor}
\usepackage{circuitikz}
\usepackage{graphicx}

\newcommand{\ra}{$\rightarrow$}
\newenvironment{6mini}{
  \begin{minipage}{6cm}
}{
  \end{minipage}
}

\title{\texttt{Boolean Algebra, K Map}\\\hrulefill}
\author{Module 5}
\date{\small{9/18/2023}}

\begin{document}
    \maketitle

    \section{Boolean algebra}
        Logic is represented by boolean equations. As with any equation, there exists simplification.
        \begin{itemize}
            \item allows for simplification methods
            \item Leads to smaller logic cicuits
        \end{itemize}
        
        \subsection*{Laws of AND \& OR}
            $xx'=0$~~~~~$x1=x$~~~~~$x0=0$~~~~~$xx=x$\\
            $x+x'=1$~~~~~$x+1=1$~~~~~$x+0=x$~~~~~$x+x=x$
        
      \begin{6mini}\vspace*{15pt}  
        \subsubsection*{Communitive}
          x+y=y+x\\xy=yx
      \end{6mini}
      \begin{6mini}\vspace*{15pt}
        \subsubsection*{Associative}  
          x + (y + z) = (x + y) + z\\x(yz) = (xy)z
      \end{6mini}
      \begin{6mini}
        \subsubsection*{distributive}  
          \[x(y+z) = xy + xz\] \[(x+y)*(x+z)\] \[xx+xz+xy+yz\] \[x(1+1z+1y)+yz\] \[x+yz\]
      \end{6mini}

      \subsubsection*{Absorption}
        \[X+XY=X+Y\] \[X'+XY=X'+Y\]

      \section*{De Morgan's Law}
        \[\bar{AB}=\bar{A}+\bar{B}\]
\end{document}